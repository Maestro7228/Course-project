\documentclass[russian,utf8,nocolumnxxxi,nocolumnxxxii]{eskdtext}
\usepackage[T1,T2A]{fontenc}
\usepackage[utf8]{inputenc}
\usepackage{amssymb,amsmath}
\usepackage{float}
\usepackage{tikz}
\usepackage{rotating}
\usepackage{graphicx}
\graphicspath{{pictures/}}
\DeclareGraphicsExtensions{.pdf,.png,.jpg}
\usepackage{pgfplots}
\usepackage{lipsum}
\usepackage{nccmath}
\usepackage{siunitx}
\usepackage[european,cuteinductors,smartlabels]{circuitikz}
\usepackage[backend=biber]{biblatex}

\begin{document}

\\{\bfСодержание}
\\1. Цель и тема курсовой работы
\\2. Задание на курсовую работу
\\3. Введение
\\4. Исследование функции
\\5. Исследование кубического сплайна
\\6. Задача оптимального распределения неоднородных ресурсов
\\7. Список литературы
\newpage

 {\large\bf 1. Цель и тема курсовой работы}

\\{\bfЦель курсовой работы:} уметь применять персональный компьютер и математические пакеты прикладных программ в инженерной деятельности.
\\{\bfТема курсовой работы:} решение математических задач с использованием математического пакета «SciLab» и системы компьютерной алгебры «Reduce».

\newpage
{\large\bf2. Задания на курсовую работу}
\\1. Даны функции $f(x)=\sqrt{3}sin(x)+cos(x),g(x)=cos(2x+\frac{\pi}{3})-1$
\\а)Решить уравнение f(x)=g(x).
\\б)Исследовать функцию h(x)=f(x)-g(x) на промежутке $[0;\frac{5\pi}{6}]$
\\2. Найти коэффициенты кубического сплайна, интерполирующего данные, представленные в векторах:\\
$V_{x}=[0,1.25,2,2.625,4.25]$
$V_{y}=[2,1.925,2.4,2.7,3.64]$\\
Построить на графике функции f(x),полученную после нахождения коэффициентов кубического сплайна. \\
Представить графическое изображение результатов интерполяции исходных данных различными методами с использованием встроенных функций splin(x,y,“natural”), splin(x,y,“clamped”), splin(x,y,“not\_a\_knot”), splin(x,y, “fast”), splin(x,y,“monotone”), interp(xx,x,y,d)\\
3. Решить задачу оптимального распределения неоднородных ресурсов.
Требуется решить следующую задачу оптимального распределения неоднородных ресурсов. Пусть в распоряжении завода железобетонных изделий (ЖБИ) имеется m видов сырья (песок, щебень, цемент) в объемах ${\bf a_i}$  .Требуется произвести продукцию {\bf n} видов. Дана технологическая норма $c_ij$  требления отдельного i-го вида сырь для изготовления единицы продукции каждого j-го вида. Известна прибыль $\pi_j$  получаема от выпуска единицы продукции j-го вида. Требуется определить, какую продукцию и в каком количестве должен производить завод ЖБИ, чтобы получить максимальную прибыль.

\begin{figure}[H]
\begin{center}
\begin{minipage}[h]{0.65\linewidth}
\center{\includegraphics[width=1\linewidth]{1.png}}  \\
\end{minipage}
\end{center}
\end{figure}


\newpage

{\bf3. Введение}

В настоящее время при решении различных как прикладных инженерных, так и чисто исследовательских задач, возникает необходимость в использовании широкого круга алгоритмов из множества разделов математики. Между тем самостоятельная реализация многих алгоритмов на некотором языке программирования может быть сложна и избыточна. Вследствие этого широкое распространение получили математические пакеты и системы компьютерной алгебры, такие как: MatLab, Octave, SciLab, Mathematica, Reduce, Mapple, призванные избавить пользователя от рутинных процедур, предоставить удобный интерфейс взаимодействия с уже написанным программным кодом и быстрым созданием нового. К сожалению, некоторые из перечисленных выше математических пакетов, будучи коммерческими по природе, имеют пакетом SciLab и системой компьютерной алгебры Reduce.

\newpage
{\bf4. Исследование функции}
\\1. Даны функции $f(x)=\sqrt{3}sin(x)+cos(x),g(x)=cos(2x+\frac{\pi}{3})-1$
\\а)Решить уравнение f(x)=g(x).
\\б)Исследовать функцию h(x)=f(x)-g(x) на промежутке $[0;\frac{5\pi}{6}]$\\
{\bfРешение уравнения.}\\
Задача а) эквивалентна следующей - требуется найти корни уравнения:\\
$h(x)=\sqrt{3}sin(x)+cos(x)-cos(2x+\frac{\pi}{3})-1$\\
Обычно при использовании мат. пакетов решение нелинейных уравнений можно получить двумя путями – численно и аналитически. Поскольку в «SciLab» с помощью стандартных функций можно получить только численное решение, при нахождении аналитического воспользуемся системой компьютерной алгебры «Reduce».\\
{\bfОтыскание численного решения.}\\
Для отыскания численного решения воспользуемся стандартной функцией «SciLab» fsolve.\\
Очевидно, что функция h(x), являющаяся линейной комбинацией периодических функций, будет иметь период равный наименьшему общему кратному периодов этих функций, то есть $T_h=HOK(T_f,T_g)=HOK(2\pi,\pi)=2\pi.$ Таким образом, достаточно численно отыскать корни на отрезке $[0,2\pi]$ и получить периодическое решение. \\
Поскольку функция fsolve основана на методе Ньютона, требуется задать начальную точку или интервал для поиска корней. С целью отыскания начальных точек построим график функции h(x) на данном отрезке:\\
function y=h(x)\\
y=sqrt(3)*sin(x)+cos(x)-cos(2*x+\%pi/3)+1\\
endfunction\\
plot(0:0.01:2*\%pi,h)\\
Полученный график изображен на Рис.1.
\newpage

\begin{figure}[H]
\begin{center}
\begin{minipage}[h]{0.65\linewidth}
\center{\includegraphics[width=1\linewidth]{2.png}}  \\
\frametitle{ Рис 1. График функции h(x)}
\end{minipage}
\end{center}
\end{figure}

Исходя из вида графика можно предположить о наличии трех или четырех корней (в окрестности точки x = 4.2 функция предположительно может дважды переходить через ноль).
\\Используя полученное знание о поведении функции воспользуемся функцией fsolve:
\\$[x,v] = fsolve(x0,f)$, где:
\\x0 – вектор начальных значений для итеративного алгоритма отыскания нулей
\\f – функция, для которой осуществляется поиск нулей
\\x – вектор нулей функции, полученных при работе алгоритма из точек x0
\\v – вектор значений функции в точках x
\\Для проверки предположения о четырех корнях укажем две начальных точки поиска с разных сторон от локального максимума, находящегося около x= 4.2
\\Листинг кода:
\\x0 = [3, 3.9,4.5,5.5];
\\$[x,v] = fsolve(x0,h)$
\newpage
v =\\
-2.220D-16 0. 0. 7.772D-16\\
x =\\
2.6179939 4.1887902 4.1887902 5.759865

Полученные точки изображены на Рис.2.
\begin{figure}[H]
\begin{center}
\begin{minipage}[h]{0.65\linewidth}
\center{\includegraphics[width=1\linewidth]{3.png}}  \\
\frametitle{Рис 2. График функции h(x) с корнями, найденными численно}
\end{minipage}
\end{center}
\end{figure}

\\Анализируя полученные значения v можно заметить, что два из них не являются нулевыми, так как решения fsolve находятся с некоторой заданной степенью точности. Также функция действительно имеет только три корня на отрезке $[0,\pi]$..

Предполагая, что корни линейной комбинации таких функций, как sin и cos могут быть кратны pi, разделим решение на pi:

x/\%pi =

0.8333333 1.3333333 1.3333333 1.8333333

Полученные десятичные дроби напоминают о числах кратных 1/3. Разделим на это число:

3*x/\%pi =
\\2.5 4. 4. 5.5

\newpage
Теперь очевидно, что корни уравнения можно записать в следующей форме:
$$x_1=\frac{5}{6}*\pi+2n\pi,n\in Z$$
$$x_2=\frac{8}{6}*\pi+2n\pi,n\in Z$$
$$x_3=\frac{11}{6}*\pi+2n\pi,n\in Z$$
Таким образом, путем нехитрых манипуляций на основе численного решения было получено аналитическое.\\
В случае, если бы, описанное выше преобразование ускользнуло от нашего внимания мы получили бы следующее решение:
$$x_1=2.6179939+2n\pi,n\in Z$$
$$x_2=4.1887902+2n\pi,n\in Z$$
$$x_3=5.759865+2n\pi,n\in Z$$
Причем из-за периодического характера функции погрешность относительного истинного значения корня находилась бы в заданных в функции fsolve пределах.
\newpage
Отыскание аналитического решения.

Для отыскания аналитического решения воспользуемся функцией solve из системы компьютерной алгебры «Reduce»:

solve(expr,var); где

expr – список из уравнений (то есть система)

var – список из переменных, относительно которых решаются уравнения expr

При попытке разрешить уравнение h(x)= 0 относительно x:\\
solve(sqrt(3)sin(x)+cos(x)-cos(2x+pi/3)-1,x);\\
получаем
\begin{figure}[H]
\begin{center}
\begin{minipage}[h]{0.65\linewidth}
\center{\includegraphics[width=1\linewidth]{4.png}}  \\
\frametitle{}
\end{minipage}
\end{center}
\end{figure}

То есть решение данного уравнения не было найдено.\\
Упростим данное уравнение, воспользовавшись двумя тригонометрическими тождествами: $$sin(x+y)=sin(x)cos(y)+cos(x)sin(y)$$
$$cos(2x)=1-2sin^2(x) $$
$$\sqrt{3}sin(x)+cos(x),g(x)-cos(2x+\frac{\pi}{3})+1$$ $$=2(sin(x)cos(\frac{\pi}{6})+cos(x)sin(\frac{\pi}{6}))+2sin^2(x+\frac{\pi}{6})$$ $$=2(sin(x+\frac{\pi}{6})+sin^2(x+\frac{\pi}{6})$$\\
и получим тривиальное уравнение, эквивалентное исходному
$$2(sin(x+\frac{\pi}{6})+sin^2(x+\frac{\pi}{6})=0$$
Применим к нему функцию solve:\\
solve(2sin(x+pi/6)*(1+sin(x+pi/6)));
и получим решение:
\newpage

\begin{figure}[H]
\begin{center}
\begin{minipage}[h]{0.65\linewidth}
\center{\includegraphics[width=1\linewidth]{5.png}}  \\
\frametitle{}
\end{minipage}
\end{center}
\end{figure}


где arbint (arbitrary integer) является произвольным целым числом. Запишем решение в более привычной форме:

$$x_1=\frac{5}{6}*\pi+2n\pi,n\in Z$$
$$x_2=-\frac{1}{6}*\pi+2n\pi,n\in Z$$
$$x_3=\frac{8}{6}*\pi+2n\pi,n\in Z$$
$$x_4=-\frac{4}{6}*\pi+2n\pi,n\in Z$$\\
Периодические решения для $x_3$и$x_4$совпадают, а периодическое решение для $x_2$ можно записать в виде:
$$x_2=\frac{11}{6}*\pi+2n\pi,n\in Z$$\\
Таким образом, применив два различных подхода, мы отыскали один и тот же набор корней нелинейной функции
\newpage
{\bfИсследование функции на заданном промежутке.}\\
Необходимо исследовать функцию h(x)  на промежутке $[0,5\frac{\pi}{6}]$.
Частично функция была исследована в предыдущем разделе. В частности на заданном отрезке функция имеет ровно один корень в точке $x=5\frac{\pi}{6}$.\\
Проведем дальнейшее исследование функции с помощью системы «Reduce», как более располагающей к аналитическому изучению функции и её производных. Для начала определим функцию h в уже упрощенном виде в пространстве имен:\\
h(x):=2sin(x+pi/6)*(1+sin(x+pi/6));
Определим значение функции на концах отрезка с помощью оператора подстановки sub:
sub(exp,f)=g, где\\
g-результат, полученный при подстановке списка алгебраических выражений exp в функцию f.\\
Очевидно, подстановка выражения вида x = const в функцию, зависящую только от аргумента x,эквивалентна её вычислению в точке const.\\
Тогда:\\
sub(x=0,h) = 3/2\\
sub(x=5pi/6,h) = 0\\
Отыщем первую и вторую производную аналитически с помощью оператора df:\\
df(f,x,n) = dif, где\\
dif – аналитическая форма производной n-го порядка для функции f по переменной x.\\
Определим в пространстве имен производные первого и второго порядка и выведем их:\\
dh1:= df(h,x,1);\\
dh1:= df(h,x,2);\\
plot(dh1,x=(0..5pi/6));
\newpage
plot(dh2,x=(0..5pi/6));\\
Полученные графики для первой и второй производной представлены на Рис. 3 и Рис. 4 соответственно.
\begin{figure}[H]
\begin{center}
\begin{minipage}[h]{0.70\linewidth}
\center{\includegraphics[width=1\linewidth]{6.png}}  \\
\frametitle{Рис. 3. График первой производной функции h(x)}
\frametitle{}
\end{minipage}
\end{center}
\end{figure}

\begin{figure}[H]
\begin{center}
\begin{minipage}[h]{0.70\linewidth}
\center{\includegraphics[width=1\linewidth]{7.png}}  \\
\frametitle{Рис. 3. График второй производной функции h(x)}
\frametitle{}
\end{minipage}
\end{center}
\end{figure}
Отыщем аналитически нули первой и второй производной, используя оператор solve:\\
solve(dh1,x);
\newpage
\begin{figure}[H]
\begin{center}
\begin{minipage}[h]{0.70\linewidth}
\center{\includegraphics[width=1\linewidth]{8.png}}  \\
\frametitle{}
\frametitle{}
\end{minipage}
\end{center}
\end{figure}
solve(dh2,x);

\begin{figure}[H]
\begin{center}
\begin{minipage}[h]{0.70\linewidth}
\center{\includegraphics[width=1\linewidth]{9.png}}  \\
\frametitle{}
\frametitle{}
\end{minipage}
\end{center}
\end{figure}

Можно заметить, что из периодических решений для первой производной только $x=\frac{\pi}{3}$ принадлежит отрезку, на котором исследуется функция. То есть на данном отрезке первая производная имеет один ноль.\\
Аналитическая форма для второй производной уже не так проста для восприятия, поэтому воспользуемся оператором нахождения численного решения num\_solve, аналогичного функции fsolve в «SciLab»:\\
num\_solve(f,x = const) = f0, где\\

f0 – ноль функции f, найденный численно при начальной точке работы алгоритма x = const.

В качестве начальных точек работы алгоритма, основываясь на виде графика второй производной, зададим x = 0.2 и x = 2.4:

num\_solve(dh2,x = 0.2) = 0.111

num\_solve(dh2,x = 2.4) = 1.983

Основываясь на полученных результатах можно сказать, что функция:

1) Возрастает на (0,$\frac{\pi}{3}$)

2) Убывает на ($\frac{\pi}{3},5\frac{\pi}{6}$)

3) Имеет глобальный максимум в точке

4) Имеет глобальный минимум в точке $x=5\frac{\pi}{6}$

5) Точки перегиба x=0.111 и x=1.193

6) Выпукла вверх на (0,0.111)∪(1.193,$5\frac{\pi}{6}$)

7) Выпукла вниз на (0.111,1.193)
\newpage
{\bf5. Исследование кубического сплайна.}\\
Найти коэффициенты кубического сплайна, интерполирующего данные, представленные в векторах:\\
$V_x=[0,1.25,2,2.625,4.25]$
 $V_y[2,1.925,2.4,2.7,3.64]$\\
 Построить на графике функцию f(x), полученную после нахождения коэффициентов кубического сплайна. Представить графическое изображение результатов интерполяции исходных данных различными методами с использованием встроенных функций splin(x,y,“natural”), splin(x,y,“clamped”), splin(x,y,“not\_a\_knot”), splin(x,y, “fast”), splin(x,y,“monotone”), и interp(xx,x,y,d)
 Оценить погрешность интерполяции в точке x = 3.1. Вычислить значение функции в точке x = 2.1.\\
{\bf Нахождение коэффициентов кубического сплайна.}\\
Найдем коэффициенты кубическом сплайна, следующим образом: получив численное значение производных в точках с помощью функции splin, отыщем 4 (по количеству интервалов между узлами интерполяции) решения для системы линейных уравнений, в которых переменными являются коэффициенты сплайна:
$A_i*Cf_i=Y_i$\\
i=0,...,4-индекс интервала\\
$A_i=$
\begin{bmatrix}
1& x_i&x^2_i&x^3_i\\
1& x_i_+_1&x_i_+_1^2&x_i_+_1^2\\
0&1&2x_i&3x^2_i\\
0&1&2x_i_+_1&3x_i_+_1^2\\

\end{bmatrix}\\
$Y_i=(y_i,y_i_+_1,d_i,d_i_+_1)^T$\\
$x_i$-узлы интерполяции\\
$y_i$-значение функции в узлах интерполяции
$d_i$-значение производной в углах интерполяции
$Cf_i$-коэффициенты i-го кубического сплайна
\newpage
$Cf_i=A_i^-1*Y_i$

Код соответствующий решению уравнений (выбранное граничное условие – равенство третьих производных слева и справа для точек x_2 и x_3):

x = [0,1.25,2,2.625,4.25];

y = [2,1.925,2.4,2.7,3.64];

d = splin(x,y);

for i = 1:4

q = x(i);

w = x(i+1);

Cf (:,i) = [1,q,q^2,q^3;1,w^2,w^3;0,1,2*q,3*q^2;0,1,2*w,3*w^3] / [y(i);y(i+1),d(i),d(i+1)]

end\\
Cf =\\
\begin{bmatrix}
2.& 2.&0.3966184&-0.3966184\\
-1.0107514& -1.0107514&2.5841762&2.5841762\\
1.01931902&1.01931902&-0.7781536&-0.7781536\\
-0.2069672&-0.2069672&0.0926101&0.0926101\\
\end{bmatrix}\\
 Построим график интерполянта и интерполянта на линейных сплайнах итеративно:\\
for i=1:n\\
t = linspace(x(i),x(i+1));\\
plot(t,interpln([x;y],t,”black”);\\
plot(t,cfs(:,i)'*[ones(t); t; t.^ 2; t.^3]);\\
end
\newpage

\begin{figure}[H]
\begin{center}
\begin{minipage}[h]{0.70\linewidth}
\center{\includegraphics[width=1\linewidth]{10.png}}  \\
\frametitle{Рис. 3. График интерполянта, полученный из коэффициентов кубических сплайнов и интерполянта из линейных сплайнов}
\end{minipage}
\end{center}
\end{figure}
Найдем значение в точке x = 2.1, используя коэффициенты третьего кубического сплайна:\\
xp = 2.1;\\
yp = Cf(:,3)’*[1,xp, xp^2, xp^3];\\
yp =\\
2.4561559\\
Оценим погрешность интерполяции относительно линейного сплайна в точке x = 3.1:\\
xr = 3.1;\\
yl = interpln([x,y],xr);\\
yp = Cf(:,4)’*[1,xr, xr^2, xr^3];\\
Err = abs(yp-yl)\\
Err =\\
0.0795513\\
\newpage
{\bfИнтерполяция встроенными методами.}\\
В математическом пакете «SciLab» можно провести интерполяцию пользуясь парой команд:\\
d = splin(x,y,”method”);\\
is = interp(xx,x,y,d);\\
Где $x=[x_1,x_1,...,x_n_-_1,x_1]$\\
y – значения функции в узлах интерполяции\\
is – значения интерполянта (кубического сплайна,\\ интерполирующего заданную функцию) вычисленные в точках xx .\\
”method” – параметр, отвечающий за граничное условие, налагаеме на интерполянт\\
Граничные условия, соответствющие различным параметрам:\\
1)”natural”-производные в точках x1,xn интерполянты равны нулю\\
2)”clamped”-явное задание производных в точках $x_1,x_n$\\
3)”not\_a\_knot”-третья производная слева и справа равна для точек $x_2$,$x_n_-_1$\\
4) ”fast” – «быстрый» расчет сплайна на основе обычной интерполяции кубическим полиномом\\
5) ”monotone” – на интервалах между узлами интерполяции интерполянт является монотонным\\
Для построения графиков интерполянтов, полученных различными методами будем применять код общего вида, подставляя нужный параметр:
xx = 0:0.01:4.25;\\
x = [0,1.25,2,2.625,4.25];\\
y = [2,1.925,2.4,2.7,3.64];\\
d = splin(x,y,”parameter”);\\
is = interp(xx,x,y,d);\\
plot(xx,is);\\
plot(x,y,”red o”);\\

\newpage
\begin{figure}[H]
\begin{center}
\begin{minipage}[h]{0.70\linewidth}
\center{\includegraphics[width=1\linewidth]{11.png}}  \\
\frametitle{Рис 4. Интерполянт, полученный с помощью splin(,,”natural”}
\end{minipage}
\end{center}
\end{figure}

\begin{figure}[H]
\begin{center}
\begin{minipage}[h]{0.70\linewidth}
\center{\includegraphics[width=1\linewidth]{12.png}}  \\
\frametitle{Рис 5. Интерполянт, полученный с помощью splin(,,”clamped”)с производными на концах (pi/3,pi/3)}
\end{minipage}
\end{center}
\end{figure}

\begin{figure}[H]
\begin{center}
\begin{minipage}[h]{0.70\linewidth}
\center{\includegraphics[width=1\linewidth]{13.png}}  \\
\frametitle{Рис 6. Интерполянт, полученный с помощью splin(,,”not_a_knot”)}
\end{minipage}
\end{center}
\end{figure}
\newpage

\begin{figure}[H]
\begin{center}
\begin{minipage}[h]{0.65\linewidth}
\center{\includegraphics[width=1\linewidth]{14.png}}  \\
\frametitle{Рис 7. Интерполянт, полученный с помощью splin(,,”clamped”) с производными на концах(-pi/2,pi/3)}
\end{minipage}
\end{center}
\end{figure}

\begin{figure}[H]
\begin{center}
\begin{minipage}[h]{0.65\linewidth}
\center{\includegraphics[width=1\linewidth]{15.png}}  \\
\frametitle{Рис 8. Интерполянт, полученный с помощью splin(,,”fast”)}
\end{minipage}
\end{center}
\end{figure}

\begin{figure}[H]
\begin{center}
\begin{minipage}[h]{0.65\linewidth}
\center{\includegraphics[width=1\linewidth]{16.png}}  \\
\frametitle{Рис 9. Интерполянт, полученный с помощью splin(,,”monotone”)}
\end{minipage}
\end{center}
\end{figure}
\newpage
{\bf6. Задача оптимального распределения неоднородных ресурсов.}\\
Требуется решить следующую задачу оптимального распределения неоднородных ресурсов. Пусть в распоряжении завода железобетонных изделий (ЖБИ) имеется m видов сырья (песок, щебень, цемент) в объемах ${\bf a_i}$  .Требуется произвести продукцию {\bf n} видов. Дана технологическая норма $c_ij$  требления отдельного i-го вида сырь для изготовления единицы продукции каждого j-го вида. Известна прибыль $\pi_j$  получаема от выпуска единицы продукции j-го вида. Требуется определить, какую продукцию и в каком количестве должен производить завод ЖБИ, чтобы получить максимальную прибыль.\\
Исходные данные:\\
\begin{figure}[H]
\begin{center}
\begin{minipage}[h]{0.65\linewidth}
\center{\includegraphics[width=1\linewidth]{1.png}}  \\
\end{minipage}
\end{center}
\end{figure}
Так как данная задача является целочисленной задачей линейного программирования (ILP), стандартная функция мат. пакета «SciLab» для решения задач линейного программирования karmarkar(…)не даст верного решения, если оптимальное решение для соответствующей задачи без целочисленного ограничения не является целочисленным или «близким» к нему.

Для решения задачи воспользуемся функций lp_solve из пакета lpsolve:

[x,f] = lp\_solve(c,A,b,e,vlb,[],xint), где:

A – матрица значений технологической норм

b – вектор ограничений на объем используемого сырья

c – вектор значений целевой функции - прибыли (значения вектора положительны, так как данная функция решает задачу максимизации целевой функции)

e – вектор, определяющий оператор отношения для ограничений (≤, ≥, =)

vlb – вектор, задающий нижнюю границу переменных решения

xint – вектор, задающий целочисленное ограничение на переменные

x – вектор решения, доставляющий максимум целевой функции

Листинг кода:

A = [9,5,2,9;10,8,3,5;9,9,1,8];

b = [18,15,20]’;

c = [40,60,20,25];

e = [-1,-1,-1];

vlb = [0,0,0];

xint = [1,2,3,4];

[x,f] = lp\_solve(c,A,b,e,vlb,[],xint)

x =

0.

1.

2.

0.

f =

100.

Таким образом, искомым целочисленным решением доставляющим максимум целевой функции является вектор [0;1;2;0], а значением целевой функции, отвечающему этому вектору, - 100.

Для достижения максимальной прибыли в сто условных единиц предприятию необходимо произвести одну единицу изделия №2 и две единицы изделия №3.
\newpage
{\bf 7. Выводы}

Были изучены встроенные функции математического пакета «SciLab» и операторы системы компьютерной алгебры «Reduce». Полученные знания были применены при решении задач: нахождения нулей функции, её аналитического исследования, интерполяции кубическими сплайнами функции от одной переменной, целочисленного линейного программирования.
\newpage
{\bf8. Список литературы}
\\1. Reduce. User’s manual
\\2. Introduction in SciLab
\\3. Optimization in SciLab
\end {document}
